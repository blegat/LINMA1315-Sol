\documentclass[11pt,a4paper]{article}

% French
\usepackage[utf8x]{inputenc}
\usepackage[frenchb]{babel}
\usepackage[T1]{fontenc}
\usepackage{lmodern}
\usepackage{url}

% Math symbols
\usepackage{amsmath}
\usepackage{amssymb}
\usepackage{amsthm}
\usepackage{subfigure} %Allows to have several figures on the same line.
\usepackage{hyperref} %Allows to make references (\ref{}), pdf links are now clickable.
\usepackage{fmtcount} %Allows to use counters
\usepackage{fourier-orns} % Allows to display the \danger symbol
\usepackage{here} % Allows to place a figure where we want

\usepackage{makeidx} %Allows to create an index
\usepackage{enumerate} %Used where??
\makeindex
\usepackage[totoc]{idxlayout} %Allows to add the index in the table of contents

\usepackage{todonotes}


\newcommand{\R}{\mathbb{R}}
\renewcommand{\L}{\mathcal{L}}

% cfr. APE
\newcommand{\cfrape}[2]{Voir l'exercice~#2 de l'ape~#1}

\newcommand{\nosolution}
{Cet exercice ne contient pas encore de solution.
Vous êtes invité à nous en soumettre une à l'adresse suivante
\begin{center}
\url{https://github.com/blegat/LINMA1315-Sol}
\end{center}
ou par mail.}

\title{Principes d'analyse fonctionnelle: Solution des exercices}
\author{Benoît Legat}

\begin{document}
\maketitle

Ce document comprend les solutions de \cite{willem2008principes}
utilisé à l'UCL dans le cours LINMA1315 donné par Michel Willem.
Certains des exercices sont résolus en séance par Alban Jago
et nous avons déjà un correctif, leur solution est donc omise.

\section{Distances}
\begin{enumerate}
  \item \cfrape{1}{4}.   %  1
  \item \cfrape{1}{5}.   %  2
  \item \cfrape{3}{1d}.  %  3
  \item \cfrape{3}{1ab}. %  4
  \item $w_k$ est ouvert par la proposition~2.12.
    $w_k$ est dense car $\mathbb{Q} \subset w_k$ et $\mathbb{Q}$ est dense dans $\mathbb{R}$.
    On peut être plus complet avec une preuve par l'absurde.
    Si $w_k$ n'est pas dense dans $\mathbb{R}$,
    il existe $x \in \mathbb{R}$ et $\exists r > 0$ tels que
    $w_k \cap B(x,r) = \emptyset$.
    Seulement, comme $\mathbb{Q}$ est dense dans $\mathbb{R}$, $B(x,r) \cap \mathbb{Q} \neq \emptyset$.
    Comme $\mathbb{Q} \subset w_k$, c'est absurde.
  \item \label{ex:unicompact}
    \footnote{On remarque que comme la suite est décroissante, $\cap_{n=1}^\infty K_n$ serait en fait $\lim_{n\to\infty} K_n$
    si on définissait une distance pour les ensembles.}
    La difficulté de cet exercice est de faire apparaitre une suite pour pouvoir commencer à utiliser
    le fait que les $K_n$ soient compacts et que $U$ soit ouvert.
    Supposons par l'absurde que pour tout $n$, $K_n \setminus U \neq \emptyset$.
    On peut alors prendre une suite $u_n$ avec $u_n \in K_n \setminus U$.
    Comme la suite est décroissante $u_n \in K_1$ pour tout $n$.
    Comme $K_1$ est compact, $u_n$ possède une sous-suite convergente $u_{n_k}$.
    Soit $u \in K_1$ la limite de cette sous-suite.
    En faisant le même raisonnement pour chaque $K_i$ avec la suite partant de $u_i$,
    on trouve à chaque fois le même $u$ et on en conclut que $u \in K_n$ pour tout $n$
    et donc que $u \in \cap_{n=1}^\infty K_n$.

    Seulement, par la définition de $u_n$, $u_n$ appartient au complémentaire de $U$.
    Mais par la proposition~2.11, ça siginifie que $u$ appartient également au complémentaire de $u$.
    $u$ n'appartient donc pas à $U$ alors qu'il appartient à $\cap_{n=1}^\infty K_n$,
    ce qui est absurde car $\cap_{n=1}^\infty K_n \subseteq U$.
  \item Idem que pour l'exercice~\ref{ex:unicompact},
    on prend $u_n \in K \setminus U_n$.
    Comme $K$ est compact, $u_n$ possède une sous-suite qui converge vers $u \in U$.
    $u_n$ appartient au complémentaire de $U_n$ et
    par la proposition~2.11, le complémentaire de de $U_n$ est fermé donc la limite
    de la sous-suite de $u_n$ converge hors de $U_n$.
    Ce qui signifie que $u$ appartient à $K$ et pas à $\cup_{n=1}^\infty U_n$ ce
    qui est absurde.
  \item \cfrape{4}{4}.   %  8
  \item On ne peut pas appliquer Dini car les $u_n$ ne sont pas nécessairement continues.
    Prenons contre-exemple où
    \[ u_n(x) =
      \begin{cases}
        1 & x = 0\\
        0 & 0 < x \leq \frac{1}{n}\\
        1 & \frac{1}{n} < x \leq 0.
      \end{cases}
    \]
    On voit que $u_n$ converge simplement vers $u(x) = 1$.
    Mais $\sup_{x \in X} d(u_n(x), u(x)) = 1$ quel que soit $n$
    donc $u_n$ ne converge pas uniformément vers $u$.
  \item
    \nosolution
  \item
    \nosolution
  \item
    \nosolution
  \item \cfrape{3}{4}.   % 13
  \item
    \nosolution
\end{enumerate}

\section{Intégrales}
\begin{enumerate}
  \item
    \begin{enumerate}
      \item Comme $|\cdot|$ est continu, soit la suite $x_k = k$, on a, par le lemme~3.2,
        $\lim_{k\to\infty} |u(k)| = |\lim_{k\to\infty} u(k)|$.
        Dès lors, si $u$ est tel que $\lim_{k\to\infty} u(k)$ existe,
        alors $|u|$ aussi.
      \item C'est évident car
        $\lim_{k\to\infty} \alpha u(k) + \beta v(k) = \alpha \lim_{k\to\infty} u(k) + \beta \lim_{k\to\infty} v(k)$.
      \item Soit $x_k = u(k)$. Si $u \geq 0$, on sait que $(x_k) \subseteq [0;\infty[$ qui est fermé.
        Si $(x_k)$ converge, il converge donc dans $[0;\infty[$ par la proposition~2.11.
      \item En prenant
        \[
          u_n(k) =
          \begin{cases}
            0 & k \leq n\\
            1 & k > n,
          \end{cases}
        \]
        on a bien $u \downarrow 0$ mais $\lim_{k\to\infty} u_n(k) = 1$ donc $\lim_{n\to\infty} \lim_{k\to\infty}u_n(k) = 1 \neq 0$.
    \end{enumerate}
  \item % 2
    Soit $u:[0,1]\to\mathbb{R} : x \mapsto 2x$, on a
    \[
      \min(u,1) =
      \begin{cases}
        2x & x \leq 0.5\\
        1 & x > 0.5,
      \end{cases}
    \]
    qui n'est plus linéaire.
  \item % 3
    Soit $u$ tel que $\sum_{n=0}^\infty |u(n)| < \infty$.
    en posant $u_k = u^+ \chi_{\{1,\ldots,n\}}$, on a $u_k \in \L$ car on a une nombre fini de valeurs non-nulles.
    De plus, $u_k$ est croissant car $\chi_{\{1,\ldots,n\}}$ est croissant et $u^+$ est positif.
    On a alors que $\lim_{k\to\infty} u_k = u^+ \in \L^+$.
    Par le même raisonnement, $-u^- \in \L^+$ et donc $u = u^- + u^+ \in \L^1$.
    Il reste à prouver que si $\sum_{n=0}^\infty |u(n)| < \infty$, alors $u \notin \L^1$.

    Si $\R$ était mesurable, ça voudrait dire que $\chi_\R$ est mesurable.
    Hors si $\chi_\R = \lim u_n$, alors il a une nombre dénombrable de valeurs non-nulle
    ce qui est absurde car $\R$ est non-dénombrable.
  \item % 4
    \nosolution
  \item % 5
    \nosolution
  \item % 6
    \cfrape{10}{2}.
  \item % 7
    \nosolution
  \item % 8
    \nosolution
  \item % 9
    \nosolution
\end{enumerate}

\bibliographystyle{plain}
\bibliography{biblio}

\end{document}
